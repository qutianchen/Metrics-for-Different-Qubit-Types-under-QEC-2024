\section{Discussion and Conclusion}
\label{section: conclusion}

% Discuss
%   Results
%   Limitations

% Conclusion
%   No free lunch
%   

The purpose of this report was to present an overview of the state-of-the-art for various qubit types. To this end, six qubit types were discussed according to a set of metrics. For each qubit, this report compiled the most optimistic results of recent studies and highlighted the obstacles that need to be overcome in the near future to enable FTQC.

% Results
When looking at the metrics compiled in this report, all qubit types would be feasible for FTQC. While the bosonic and photonic qubits are still mostly theoretical models, quantum computing has been realized for the other four qubits, superconducting, semiconducting, neutral atom, and trapped-ion.
These latter qubits have each, in individual experiments, been shown to be well on their way to the FTQC-era.

% Limitations
This statement, however, also highlights the main limitation of this report. Because only the most optimistic values were considered, the resulting overview is also optimistically biased in favor of these qubit types. It is therefore important to reiterate that most results mentioned in this report were obtained in isolated experiments, where the conditions would have been perfect for each specific case.
This could imply that the setup that allowed for one result to be obtained would hinder another experiment or even make it completely infeasible.

When observing the results of this report, it is clear to see that there is no single best qubit type as of yet. Some qubits may be more developed than others, but each qubit has its own set of advantages and disadvantages.
Choosing a relevant qubit depends on the use-case, in general the superconducting qubit is a relatively mature technology, semiconducting qubits can use existing manufacturing infrastructure for fast scaling, neutral atom and trapped-ion qubits are very stable, and photonic qubits appear ideal for a quantum internet.

Similar to the "no free lunch theorem" for optimization algorithms proposed by Wolpert and Macready in \cite{Wolpert_1997}, the elevated performance of a qubit in one area is usually offset by the performance over another area. Therefore, it is currently impossible to assign one qubit as the best.
