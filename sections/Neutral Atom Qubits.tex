\subsubsection{Neutral Atom Qubits}

Neutral atom qubits have seen a steadily increasing interest as a promising candidate in the search for scalable FTQC. The fact that neutral atoms are identical qubits, have relatively long coherence times of $\sim$40 s \cite{Barnes2022}, and do not require complex circuitry and wiring compared to other qubit types \cite{ball2024} makes them interesting to investigate. A main feature of this type of qubit is the ability to dynamically reconfigure a qubit array (i.e. shuttling) in a zoned architecture, enabling an all-to-all connectivity between qubits \cite{Bluvstein2024}, \cite{Bluvstein2022}. This array of neutral atoms is laser-cooled and can be freely manipulated using optical tweezers while preserving the coherence and entanglement of the qubits without fidelity cost \cite{Bluvstein2024}, \cite{Wintersperger2023}.

Historically, an important aspect holding neutral atoms back had been the low two-qubit gate fidelity. However, new developments in laser technology have increased the fidelity to 99.5\% \cite{Evered2023}. Combining this higher fidelity with the zoned approach and qubit shuttling has shown significant results, namely the successful execution of an error-corrected quantum processor with 48 logical qubits \cite{Bluvstein2024}, which is the highest number to date \cite{Swayne2023}. The zoned approach can be scaled to $\sim$10,000 atoms by using stronger lasers and it offers constant-space-overhead to QEC, with possible implementations being high-rate quantum concatenated codes and low-density parity (LDPC) codes \cite{sunami2024scalablenetworkingneutralatomqubits}. Nevertheless, scaling beyond this might require further advancements in optical control \cite{Bluvstein2024}, \cite{sunami2024scalablenetworkingneutralatomqubits}.

Some challenges still remain in the slow initialization, gate operation and readout times, which are limited by the physical movement of the qubits. An improvement of these times was demonstrated by introducing an architecture where the qubits are stationary and  a laser is moved around to individually address the qubits using advanced optical control techniques. The gate operation and readout is then limited by optical switching with entangling gate fidelity of 99.35\% \cite{radnaev2024universalneutralatomquantumcomputer}. 
While this technique is faster than qubit shuttling, the ability for all-to-all connectivity is lost in return. Therefore, a potential middle ground might be a hybrid implementation of both optical addressing and qubit shuttling.
Additionally, a newly developed non-destructive readout method also enables the possibility for  mid-circuit readout and error correction \cite{radnaev2024universalneutralatomquantumcomputer}. This reduces the amount of non-circuit operations, such as atom arrangement and loading, leading to faster overall operations.
