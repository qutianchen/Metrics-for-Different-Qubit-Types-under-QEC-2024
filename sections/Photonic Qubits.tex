\subsubsection{Photonic Qubits}

%Photonic qubits are an upcoming and one-of-a-kind approach to quantum computing. Since these qubits implement photons, which are always active, they perform and encode information in the infinite-dimensional Hilbert space of quantum harmonic oscillators, better known as qumodes \cite{noauthor_introduction_nodate}. These photonic 'qumodes' do not share the same characteristics as other qubits. Due to these deviating characteristics, they show great potential as a candidate for scalable and FTQC and QEC. 

Photonic qubits are a specific type of bosonic qubits, encoding information similarly but using photons as information carriers. They therefore do not require using superconducting or other systems.

The current state-of-the-art photonic qumodes are mostly based on theoretical papers that promise great potential, but some experimental papers also show and prove these advantages. 
Photonic qumodes are mainly interesting due to their extremely long coherence values \cite{yan_silicon_2021} and have good scalability, which supports quantum communication and error correction. Since photons are active at room temperature and move at the speed of light, they show high potential for fast and accurate quantum error correction \cite{nourbakhsh_quantum_2022}. Another potential upside to photonics would be that the logical gate speed could potentially be at least 25 times faster than that of superconducting qubits \cite{tezrudolph_what_2023}. 

However, there are two main issues in the current research on photonics. The first one would be due to photon loss, photons can be lost during transmission or manipulation, which can lead to errors in quantum information processing \cite{noauthor_photonic_nodate}. The second problem would be that photons can not stop moving and can not interact with each other, implying that creating a two-qubit gate is difficult to achieve. However, certain methods like LDPC codes \cite{djordjevic_photonic_2009} and Linear Optics Quantum Computing (LOQC) \cite{noauthor_photonic_nodate} are being developed and researched to overcome these problems. 


\begin{comment}
https://www.nature.com/articles/s41586-021-03290-z 
Preserve the qubit information with a fidelity of 96.2 ± 0.3 
We achieve a nondestructive detection efficiency upon qubit survival of 79 ± 3 percent and a photon survival probability of 31 ± 1 percent.
This is used to distinguish the two atomic states |0a⟩ and |1a⟩ with a fidelity of (98.2 ± 0.2) percent at a photon-number threshold of one.

https://outshift.cisco.com/blog/how-powerful-are-photonic-quantum-processors
we present the minimum optical properties required to achieve a QV (quantum Volume = how big the system can get before overwhelmed by too much noise) of 2$^10$, which is a typical value reported for common quantum computing platforms such as superconducting qubits or trapped ions. 

Our findings indicate that photonic hardware would require improvements to reduce losses to approximately 10percent (or 0.5 dB) and a squeezing rate of 18 dB for coherent-pulse qubits to achieve a QV of 2$^10$. To put numbers in perspective, the record squeezing rate ever reported in optics labs around the world is 15 dB. Also, most photon loss events occur at the waveguide interface when the signal enters the chip and is measured at the detector. A typical value of loss rate at the interface is 20 percent (or 1 dB).


https://journals-aps-org.tudelft.idm.oclc.org/prxquantum/pdf/10.1103/PRXQuantum.4.030336
In this work, we report a single-photon qubit encoded in a novel superconducting cavity with a coherence time of 34 ms, representing an order of magnitude improvement compared to previous demonstrations. We use this long-lived quantum memory to store a Schrödinger cat state with a record size of 1024 photons, indicating the cavity’s potential for bosonic quantum error correction.

Our experimental results show a single-photon relaxation time of Tc1 = 25.6 ± 0.2 ms and
a coherence time of Tc2 = 34 ± 1 ms

\end{comment}