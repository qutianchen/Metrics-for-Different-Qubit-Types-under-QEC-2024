\subsubsection{Superconductor Qubits}
Superconducting qubits play a leading role in quantum computing today \cite{acharya2024quantumerrorcorrectionsurface}. With various superconducting quantum processors available, many applications in the Noisy Intermediate-Scale Quantum (NISQ) era can be carried out \cite{IBMQuantum}. Meanwhile, a steady increase in both gate fidelity and qubit number in the past years has been witnessed for superconducting qubits \cite{IBMroadmap}. All these developments also meet the roadmaps from companies such as IBM and Google \cite{IBMroadmap}\cite{Googleroadmap}. Recently, Google has also demonstrated error correction below the surface code threshold for the first time \cite{acharya2024quantumerrorcorrectionsurface}, which sheds light on the path towards the FTQC era.

 Currently, there exist superconducting quantum processors with up to 156 qubits for commercial usage (IBM Heron) \cite{IBMQuantum} and also quantum chips with more than 1000 qubits (IBM Condor). Though high in qubit numbers, the latter one's error rate is five times higher than Heron\cite{Afifi-Sabet_2023}. 
 For the $T_1,T_2^*$ time, around 1.2 ms is achieved via a single fluxonium qubit in lab \cite{T1T2}, together with Clifford gates gate fidelity 99.991\%.
 The single-qubit gate speed can achieve 4.16 ns on a two-qubit platform with gate fidelity 99.76\% via closed-loop optimization for the pulse shape \cite{Werninghaus_Egger_Roy_Machnes_Wilhelm_Filipp_2021}.
 For two-qubit gates, the IBM Heron has 68 ns CZ gate gate with fidelity 99.7\% for a 156 qubits chip.\cite{IBMQuantum} IQM has also demonstrated their CZ gate with 99.9\% fidelity \cite{IQM}.
 For initialization, demonstration in 180 ns with high fidelity has been done for single qubit. This is accomplished by coupling to a quantum-circuit refrigerator through a resonator.
 On the other hand, it took 140 ns for a fidelity 99.5\% two-state readout and a fidelity 96.9\% three-state readout with the help of a feedforward neural network.
 Latest demonstration also showed that the working tempreture can be around 250 mK \cite{anferov2024superconductingqubits20ghz} instead of the usual 10$\sim$20 mK. But this requires higher frequency (up to 24 GHz) and lowloss niobium trilayer junctions.
 
 Fast processing speeds could be a key feature in the FTQC era. Compared to the NISQ era, coherence time for a fault tolerant qubit is not as important since the average gate error is lowered to $\sim10^{-9}$. At the same time, QEC requires a large number of gates, initialization and measurements \cite{Fowler_2012} and fast gate speed is important here. Superconducting qubits are also easy to parallelize operations by using techniques such as Superconducting Quantum Interference Device (SQUID) loops to tune resonant frequencies \cite{krantz_quantum_2019}. The 2D lattice structure for the superconducting qubits also fits many prominent error correction codes such as the surface code as it emphasizes nearest neighbour interaction and local errors \cite{Fowler_2012}. 

However, superconducting qubits also have many drawbacks. Current superconducting qubits and transmission lines are huge compared to other qubits such as spin qubits. A million superconducting qubits on a flat 2D lattice can be square meters large, which could raise problems for clock synchronization since they are running at 4-8 GHz. Heating is a problem due to the low working temperature of superconducting qubits. At 10 mK, cooling power becomes a bottleneck for the number of control and measurement lines to connect to the processor \cite{krantz_quantum_2019}.

% Superconducting qubits do not seem to have many bottlenecks in the near term. However, there still remain obstacles towards the goal of a million physical qubits or 1000 logical qubits. The size of the superconducting circuit and its transmission lines is a persisting issue. Growing qubit number would also require larger and better fridges. 
% Due to the fast gate speed, FPGA could potentially be a big helper for superconducting qubits: On the one hand, it can decrease the classical processing time for the error correction decoder. On the other hand, controlling on the same level as the quantum processor can greatly reduce the control and measurement lines for the chip and lower the cooling power requirement.