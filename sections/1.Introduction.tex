\section{Introduction}
With the current knowledge, it is nearly impossible to achieve fault-tolerant quantum computing (FTQC) without proper error correction \cite{roffe2019quantum}, \cite{devitt2013quantum}. Quantum error correction (QEC) aims to recover quantum information that is lost when conducting operations with qubits \cite{roffe2019quantum}. The field of QEC is currently broad: various error correction methods can be used, which can themselves be applied to different qubits, each with its own characteristics \cite{roffe2019quantum}. Therefore, having an overview of the current state-of-the-art QEC, together with suggestions of research directions, can help researchers determine what to focus on. \par

The aim of this report is to present an overview of the state-of-the-art QEC, as well as provide suggestions for the most promising research directions. To achieve this, a literature review was conducted. Perplexity AI \footnote{https://www.perplexity.ai/} was used as a starting point for collecting sources on the topics of interest. Then, the search was extended by examining the sources referenced in the initial papers and by performing Google Scholar \footnote{https://scholar.google.com/} queries with similar keywords. The advantages and disadvantages of six qubit types (superconductor, semiconductor spin, neutral atom, trapped-ion, photonic, and bosonic) were analyzed. This information was then used to spark a discussion about which types are the most promising for QEC. \par

The report is structured as follows. Chapter \ref{section: qec} analyses the potential of six qubit types in the context of QEC. Next, Chapter \ref{section: conclusion} compiles our suggestions and conclusions. \par


% paragraph 1: background
% - background info on the subject
% - problem that needs to be solved
% - importance

% paragraph 2: aim
% - research question/aim
% - methods
% - limitations

% paragraph 3: structure