\subsection{Continuous Variable Quantum Systems}
These quantum systems encode information in infinite-dimensional Hilbert spaces, encoding information in continuous degrees of freedom such as the quadratures of electromagnetic fields, using quantum harmonic oscillators. These are better known as qumodes \cite{noauthor_introduction_nodate}, which do not share the same characteristics as other qubits. Due to these differing characteristics, they show great potential as a candidate for scalable FTQC and QEC. 


\subsubsection{Bosonic Qubits}
Bosonic qubits are a promising approach to quantum computing, often implemented in superconducting or trapped-ion systems. They have general high fidelity ($\sim$97\%) and increased error correction performance ($\sim$95-98\%) \cite{chatterjee-2023}. Bosonic qubits come in 4 main types: cat, Gottesman-Kitaev-Preskill (GKP), dual-rail, and binomial qubits. Dual-rail and binomial qubits are promising \cite{levine-2024}, \cite{michael-2016}, but dual-rails are still mainly experimental, and binomial qubits are extremely hard to implement and scale. Therefore, they will not be mentioned anymore in this section.


Cat qubits are designed to aim for beyond the NISQ era, and are rapidly being developed by different actors, such as Amazon \cite{chamberland-2022} and Alice \& Bob \cite{alice-bob-2024}. They exhibit exponential decrease in bit-flip errors, leading to coherence times over 10 s \cite{reglade-2024}, 100 s (without quantum control) \cite{berdou-2023} and even possibly exceeding 7 min \cite{alice-and-bob-2024} in different conditions. Cat qubits also promise for low overhead with LDPC codes and a universal set of gates \cite{ruiz-2024}, combined to passive error correction due to their architecture. One of the trade-offs for their inherent bit-flip error protection is a linear increase in phase-flip errors \cite{quera}, requiring other error correction methods. Additionally, maintaining their superposition of states against decoherence requires complex techniques \cite{tucker-2024}, sometimes hard to implement. Cat qubits are also currently mainly implemented on superconducting architecture, requiring extremely low temperatures to operate (around 10 mK), hence enhancing their implementation complexity.

GKP codes originated in 2001 from a paper suggesting encoding a qubit in an oscillator \cite{gottesman-2001}. They are currently being developed by Nord Quantique \cite{nord-UANTIQUE}, the University of Tokyo \cite{konno-2024}, AWS \cite{noh-2022}, and others. They have an inherent protection against both bit-flip and phase-flip errors and require a low overhead for QEC \cite{noh-2022}. GKP codes also drastically reduce hardware requirements for creating a logical qubit, and have very recently been described as the most promising type of bosonic qubit, outperforming other bosonic codes \cite{lemonde-2024}. However, the scaling and implementation complexity of GKP codes are quite a challenge \cite{grimsmo-2021}, adding on to that the need for additional error correction despite their intrinsic error correction capabilities \cite{noh-2022}. Lastly, when one error occurs, GKP codes can efficiently correct for it in a small amount of time. Yet, if in that timespan, another error occurs, the qubit's logical state progressively deteriorates \cite{nord-quantique}.
